% Created 2023-10-11 Wed 15:45
% Intended LaTeX compiler: pdflatex
\documentclass[11pt]{article}
\usepackage[utf8]{inputenc}
\usepackage[T1]{fontenc}
\usepackage{graphicx}
\usepackage{longtable}
\usepackage{wrapfig}
\usepackage{rotating}
\usepackage[normalem]{ulem}
\usepackage{amsmath}
\usepackage{amssymb}
\usepackage{capt-of}
\usepackage{hyperref}
\author{TheKamboy}
\date{\textit{<2023-09-16 Sat>}}
\title{Ghost Team: Computer Trouble}
\hypersetup{
 pdfauthor={TheKamboy},
 pdftitle={Ghost Team: Computer Trouble},
 pdfkeywords={},
 pdfsubject={},
 pdfcreator={Emacs 29.1 (Org mode 9.7)}, 
 pdflang={English}}
\begin{document}

\maketitle
\begin{center}
\includegraphics[width=.9\linewidth]{/home/kamboy/.config/emacs/.local/cache/org/persist/47/c0de97-ef86-464c-b33a-f96246692b82-be323822b7c2141d219988a488155d5a.png}
\end{center}

\begin{quote}
Keegan and his brother, Kameron, were playing a game, when they were called to fix a issue with the server room.
They fought over who fixes it and accidentally breaks the terminal in the server room.
A portal emerges and sucks them both in.

Will they make it out? Play and find out.
\end{quote}
\section{Getting Started}
\label{sec:org59d15f0}
\subsection{Requirements}
\label{sec:org22a3ab2}
\subsubsection{Linux or Mac (untested)}
\label{sec:orgf8d6a19}
If your Linux Distro supports \hyperref[sec:org467268d]{Gum}, then you are good.

Mac is untested but I think it might work since this game is just an sh script.
\subsubsection{Gum}
\label{sec:org467268d}
Gum is a CLI for \emph{glamorous shell scripts}. I use it for this game because it has a more appealing UI, and it makes game development quicker.

\href{https://github.com/charmbracelet/gum}{Get Gum here.}
\subsection{Installation}
\label{sec:orgeaaf80e}
\begin{quote}
🔨 There are no releases yet so links leading to releases may not have anything in them.
\end{quote}
It is recommended to \href{https://github.com/TheKamboy/gt-computer-trouble/releases/latest}{install the latest version of the game}.

Unpack the .zip (or .tar.gz depending on what you installed)
\subsection{How to set up the download for playing}
\label{sec:orgb2552db}
\begin{itemize}
\item Set \texttt{gtct.sh} to be executable and run it like a program in the terminal \texttt{./gtct.sh}.
\end{itemize}

or

\begin{itemize}
\item Run it in the terminal like this \texttt{sh ./gtct.sh}.
\end{itemize}
\subsection{Other Ways to Play}
\label{sec:org0d2cc0e}
\begin{quote}
💣 The Replit version is VERY buggy, so it's recommended to play with the other options instead.
\end{quote}
\href{https://replit.com/@Kamboy123/gt-computer-trouble?v=1}{You can play on Replit!}
\section{New Name}
\label{sec:org77f73e2}
Keegan's Game has now been renamed to Ghost Team, due to request from my brother.
\end{document}